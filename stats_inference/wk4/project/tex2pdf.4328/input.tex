\documentclass[]{article}
\usepackage{lmodern}
\usepackage{amssymb,amsmath}
\usepackage{ifxetex,ifluatex}
\usepackage{fixltx2e} % provides \textsubscript
\ifnum 0\ifxetex 1\fi\ifluatex 1\fi=0 % if pdftex
  \usepackage[T1]{fontenc}
  \usepackage[utf8]{inputenc}
\else % if luatex or xelatex
  \ifxetex
    \usepackage{mathspec}
  \else
    \usepackage{fontspec}
  \fi
  \defaultfontfeatures{Ligatures=TeX,Scale=MatchLowercase}
\fi
% use upquote if available, for straight quotes in verbatim environments
\IfFileExists{upquote.sty}{\usepackage{upquote}}{}
% use microtype if available
\IfFileExists{microtype.sty}{%
\usepackage{microtype}
\UseMicrotypeSet[protrusion]{basicmath} % disable protrusion for tt fonts
}{}
\usepackage[margin=1in]{geometry}
\usepackage{hyperref}
\hypersetup{unicode=true,
            pdftitle={Statistical Inference Course Project Part 1 - A Simulation Exercise},
            pdfauthor={Randall Helms},
            pdfborder={0 0 0},
            breaklinks=true}
\urlstyle{same}  % don't use monospace font for urls
\usepackage{graphicx,grffile}
\makeatletter
\def\maxwidth{\ifdim\Gin@nat@width>\linewidth\linewidth\else\Gin@nat@width\fi}
\def\maxheight{\ifdim\Gin@nat@height>\textheight\textheight\else\Gin@nat@height\fi}
\makeatother
% Scale images if necessary, so that they will not overflow the page
% margins by default, and it is still possible to overwrite the defaults
% using explicit options in \includegraphics[width, height, ...]{}
\setkeys{Gin}{width=\maxwidth,height=\maxheight,keepaspectratio}
\IfFileExists{parskip.sty}{%
\usepackage{parskip}
}{% else
\setlength{\parindent}{0pt}
\setlength{\parskip}{6pt plus 2pt minus 1pt}
}
\setlength{\emergencystretch}{3em}  % prevent overfull lines
\providecommand{\tightlist}{%
  \setlength{\itemsep}{0pt}\setlength{\parskip}{0pt}}
\setcounter{secnumdepth}{0}
% Redefines (sub)paragraphs to behave more like sections
\ifx\paragraph\undefined\else
\let\oldparagraph\paragraph
\renewcommand{\paragraph}[1]{\oldparagraph{#1}\mbox{}}
\fi
\ifx\subparagraph\undefined\else
\let\oldsubparagraph\subparagraph
\renewcommand{\subparagraph}[1]{\oldsubparagraph{#1}\mbox{}}
\fi

%%% Use protect on footnotes to avoid problems with footnotes in titles
\let\rmarkdownfootnote\footnote%
\def\footnote{\protect\rmarkdownfootnote}

%%% Change title format to be more compact
\usepackage{titling}

% Create subtitle command for use in maketitle
\newcommand{\subtitle}[1]{
  \posttitle{
    \begin{center}\large#1\end{center}
    }
}

\setlength{\droptitle}{-2em}
  \title{Statistical Inference Course Project Part 1 - A Simulation Exercise}
  \pretitle{\vspace{\droptitle}\centering\huge}
  \posttitle{\par}
  \author{Randall Helms}
  \preauthor{\centering\large\emph}
  \postauthor{\par}
  \predate{\centering\large\emph}
  \postdate{\par}
  \date{5 October 2016}


\begin{document}
\maketitle

\subsection{Introduction}\label{introduction}

This document covers the three questions on the first part of the
Statistical Inference course project. This course is part of the Data
Science Specialization offered by Johns Hopkins University via Coursera.

This project involves using R to investigate the exponential
distribution and then compare it to the Central Limit Theorem.

In this project we have to illustrate and explain the properties of the
distribution of the mean of 40 exponentials, involving the following
three tasks:

\begin{verbatim}
1. Show the sample mean and compare it to the theoretical mean of the distribution.
2. Show how variable the sample is (via variance) and compare it to the theoretical variance of the distribution.
3. Show that the distribution is approximately normal.
\end{verbatim}

In order to do this, the course authors have provided the following
parameters to work with:

\begin{verbatim}
set.seed(1980)
lambda = 0.2
n = 40
simulations = 1000

exponential_distribution_simulation <- rexp(n,lambda)

mean_exponential_distribution = 1/lambda

standard_deviation = 1/lambda
\end{verbatim}

Building on this, let's simulate the exponentials and then calculate the
mean of the simulation:

\begin{verbatim}
#simulate the exponentials

sim_exp <- replicate(simulations,rexp(n,lambda))

#calculate the mean of exponentials

mean_exp <- apply(sim_exp,2,mean)
\end{verbatim}

Let's take things one step further and use that information to plot out
a simple histogram showing the distribution of means:

\begin{verbatim}
colfunc<-colorRampPalette(c("springgreen","royalblue")) #create a gradient color, just because it looks nicer than the block colors
hist(mean_exp,main = "1000 Simulated Exponential Means",xlab="Simulation Means",col=colfunc(10))
\end{verbatim}

\subsection{Question 1}\label{question-1}

The first question requires us to show the sample mean and compare it to
the theoretical mean of the distribution.

Here's how we calculate the sample mean:

\begin{verbatim}
sample_mean = mean(mean_exp)
\end{verbatim}

To compare this to the theoretical mean of the distribution, we can put
it on the chart we made before:

\begin{verbatim}
hist(mean_exp,main = "1000 Simulated Exponential Means",xlab="Simulation Means",col="blue",breaks=20)
abline(v=sample_mean,lwd="3",col="green",lty=4)
\end{verbatim}


\end{document}
